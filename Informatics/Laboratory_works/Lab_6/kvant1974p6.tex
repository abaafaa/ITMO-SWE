\documentclass[main.tex]{subfiles}

\begin{document}

\begin{multicols}{2}

\begin{center}
\includegraphics[width=6cm]{images/pic1.png}

Рис. 1
\end{center}

\noindent
Колебательный процесс может возникать в системе только при наличии собственной частоты $\nu_0$.
В этом случае сигнал $u_3(t)$ является простейшим, так как возбуждает колебания лишь на одной частоте.
Подобный процесс хорошо известен и представляет собой гармоническое колебание, описываемое выражением
\[
u(t) = A\sin(2\pi\nu_0 t + \varphi) = A\sin(\omega_0 t + \varphi),
\]
где $\omega_0 = 2\pi\nu_0$.

\begin{center}
\includegraphics[width=6cm]{images/pic2.png}

Рис. 2
\end{center}

\noindent
Параметры $A$, $\omega_0$ и $\varphi$ считаются постоянными величинами.
Однако сигналы $u_1(t)$ и $u_2(t)$ могут одновременно воздействовать на несколько контуров,
настроенных на различные частоты.

\par
Такое поведение можно объяснить тем, что данные сигналы эквивалентны сумме нескольких синусоидальных
колебаний с различными частотами. Это свойство строго доказал французский математик Жан Фурье.

\par
Согласно теореме Фурье, практически любую периодическую функцию с частотой $\nu_0$
можно представить в виде суммы гармоник:
\[
u(t) = \sum_{k=1}^{\infty} A_k \sin(k\omega_0 t + \varphi_k),
\]
где $A_k$ — амплитуда гармоники, $\varphi_k$ — её начальная фаза, а $k$ определяет номер гармоники.

\par
Совокупность величин $A_k$ называют спектром амплитуд.
Для наглядности спектр изображают графически, откладывая значения $A_k$ по оси ординат,
а частоты $\nu$ — по оси абсцисс.

\end{multicols}

\end{document}
